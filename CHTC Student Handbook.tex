% CHTC Student Handbook

\documentclass[10pt,letterpaper]{article}
\usepackage[utf8]{inputenc}
% Change font to Helvetica
\usepackage{helvet}
\renewcommand{\familydefault}{\sfdefault}

% TODO - Add glossary
% \usepackage{glossaries}

\usepackage{hyperref}



\begin{document}
% TITLE PAGE
\title{CHTC Student Handbook \\
    \large Edition 0.1}
\author{Dakota Chambers}
\date{\today}
\maketitle

% TABLE OF CONTENTS
\tableofcontents
\clearpage

% BEGIN BOOK
\section{Student Work 101}
    \subsection{Tasks and Workflow}
        As a student worker for the CHTC you will be responsible for a large variety
        of tasks to help keep the CHTC Cluster up and running, but there a few basic
        things you will find yourself doing most often. Our primary job is to fix
        individual machines in the cluster that go down. Most often machines can
        be brought down by kernel panics, networking issues, or hardware issues.
        To fix a machine, we take a simple approach to diagnose the issue and come
        up with a solution.

        \begin{enumerate}
            \item Diagnose the problem. You might need to plug into the machine
            with a console, or SSH into it if it is still online. If a machine
            kernel panics it will often print a memory stack trace to the console
            and freeze up. You will need to physically reboot the machine to proceed.
            \item Find a solution. Often times rebooting the machine will clear up
            minor problems such as kernel panics or networking issues.
            \begin{itemize}
                \item Kernel Panic (crash) - reboot, check server for errors.
                \item Networking issue - reboot machine or restart networking, then
                go to advanced networking troubleshooting if still needed.
                \item Hardware issue - diagnose the hardware issue and fix it.
                (Replace a bad disk, make note of a bad RAID card, etc)
            \end{itemize}
            \item Fix the problem and confirm that condor is running jobs again.
            \begin{itemize}
                \item his might mean rebooting the machine, rebuilding the machine,
                replacing a bad disk and rebuilding, or something else entirely.
                Fix it if possible. If not - make note of the issue in the ticket
                and await further instructions.
            \end{itemize}
        \end{enumerate}

        Some other tasks you might be assigned include, but are not limited to:
        archiving condor releases to DVD, building new execute machines, working
        with Dell/Cisco Tech Support to get replacement parts, moving servers, a
        nd other random tasks assigned by full time CHTC staff.
    \subsection{RT Ticket System}
        We use the Request Tracker (RT) Ticket System for managing work at the CHTC.
        \href{https://crt.cs.wisc.edu/rt/}{crt.cs.wisc.edu} is the web address to
        access our Request Tracker.This will be your home for managing work tickets.
        The basic flow is: You are assigned a ticket, click on the ticket name to
        open it. By default, your 10 highest priority, or newest, tickets will be
        listed on your home page. You can view all of your tickets by clicking the
        “10 Highest Priority Tickets I Own” link (counter-intuitive, I know).

        Once you have selected a ticket, you can read all correspondence related
        to that particular issue. Under the “actions” button (top right) the most
        common things you will do are “reply” and “resolve.” Whenever you make
        changes to a machine, make sure to log what you did in the ticket. When
        the machine is fixed or the task is completed, you can resolve the ticket.

    \subsection{Monitoring Nodes}
        You can monitor the cluster on \href{monitor0.chtc.wisc.edu}{monitor0.chtc.wisc.edu}.
        This has links to Icinga, our host monitoring application, Grafana, and Ganglia.
        All of these are extremely powerful tools that can be used to monitor the
        status of machines in the cluster and diagnose problems with individual nodes.

    \subsection{How to Use This Guide}
        \begin{itemize}
            \item Any line or section beginning with “\$” is a command line operation.
            Enter it directly as it appears, without the preceding “\$”.
            \item Any bracketed item (eg. [System Name]) needs to be replaced with
            the corresponding entity. (eg. change [System Name] $\rightarrow$
            e1000.chtc.wisc.edu).
            \item Following the step-by-step walkthroughs to complete most basic
            tasks and use this as a reference manual.
            \item The appendix includes a complete list of node locations, useful
            for location a machine for physical changes (hard reboot, HDD swap, etc).
        \end{itemize}

\clearpage

\section{Rebuilding a Node}
    Rebuilding nodes is one of the primary responsibilities of student CHTC employees.
    Here’s a step-by-step walkthrough for you.

    \begin{enumerate}
        \item Enable netboot in cobbler.
        \begin{enumerate}
            \item Via Command Line:
            \begin{enumerate}
                \item SSH into wid-service-1.chtc.wisc.edu
                \item {\fontfamily{cmtt}\selectfont \$ sudo cobbler system edit
                --name=[server name] --netboot-enabled=True }
                \item Check that netboot is enabled $\rightarrow$
                {\fontfamily{cmtt}\selectfont \$ sudo cobbler system report
                --name=[server name] }
                \item Sync Cobbler $\rightarrow$ {\fontfamily{cmtt}\selectfont
                \$ sudo cobbler sync }
            \end{enumerate}
            \item Via Web Interface
            \begin{enumerate}
                \item Open \href{https://wid-service-1.chtc.wisc.edu/cobbler_web}{Cobbler}
                in your web browser
                \item Click on “Systems” under the Configuration tab on the left side.
                \item Find the node you want to rebuild and click on it’s name in
                the list to open the node configuration tool.
                \item Alternatively, navigate directly to \\
                wid-service-1.chtc.wisc.edu/cobbler\_web/system/edit/[node name]
                \item Click on the “general” drop down button to reveal more options
                \item Check the “enable netboot” option
                \item Hit “save” on the bottom of the page
                \item When you are redirected to the cobbler system list page,
                hit “Sync” Under the Actions tab on the left side.
                \item Once you get a popup notification on the top-right of the
                screen, the sync has complete. This may take a few seconds.
            \end{enumerate}
            \item Make sure you have the correct profile enabled in Cobbler as
            well. If you are doing a standard rebuild you probably won’t have to
            change it, but make sure if it’s a multi-disk execute node that it is
            set to the correct SL66 Exec profile.
        \end{enumerate}
        \item Reboot the machine and netboot it.
        \begin{enumerate}
            \item Most machines are set to netboot by default, meaning if you
            reboot them they will search for a netboot entry and if they find it,
            they will netboot automatically. If a machine is not netbooting automatically,
            you may need to press a button on the keyboard (Often F12) when it POSTS
            in order to force it to netboot. If you have tried these and it still
            won’t boot from the network, go into the BIOS and change the boot order
            to set netboot as boot priority \#1.
            \item If done correctly, it should launch the Scientific Linux installer.
            Once you see this is happening, move on to the next step.
        \end{enumerate}
        \item Run Puppet.
        \begin{enumerate}
            \item We are going to need to run puppet once the machine is rebuilt
            in order to configure it.
            \item While the machine is rebuilding, SSH into wid-service-1.chtc.wisc.edu
            \item Run this command: {\fontfamily{cmtt}\selectfont \$ sudo puppetca
            -c [node name] } on wid-service-1.
            \item If you do this before the machine finishes rebuilding, it may
            run puppet automatically when it rebuilds. To see if it is doing this:
            When the machine is booting Scientific Linux, hit an arrow key
            $\leftarrow$ or $\rightarrow$ on the console keyboard to view the boot
            log. If the log is paused on “{\fontfamily{cmtt}\selectfont Starting:
            anamon…     [OK] }” for a while, that mean’s it’s running puppet. Good
            job! Once it finishes booting now, you should be able to log in with your username.
            \item If the machine does not automatically run puppet, connect a console
            to the machine and log in as a root user (ask Admin for root login)
            \item After clearing the puppet files from wid-service-1 in the previous
            steps, run {\fontfamily{cmtt}\selectfont \$ sudo rm -rvf /var/lib/pup/ssl }
            on the target node. DO NOT RUN THIS ON wid-service-1.
            \item Then run {\fontfamily{cmtt}\selectfont \$ sudo puppetd -tv
            --configtimeout=1000 }
            \item Note: Puppet will not run if networking is broken (try to restart
            networking or reboot the machine if this is the case) or the system
            clock is broken. To set the clock run: {\fontfamily{cmtt}\selectfont
            \$ rdate -s ntp.doit.wisc.edu }
        \end{enumerate}
        \item Log in and confirm condor is running.
            \begin{enumerate}
                \item Once puppet finishes running, you should be able to log in
                with your own username
                \item Confirm condor is running: {\fontfamily{cmtt}\selectfont \$
                condor\_status \$HOSTNAME }
                \item You should also see Condor running with {\fontfamily{cmtt}\selectfont
                \$ ps aux }
            \end{enumerate}
    \end{enumerate}
\clearpage

\section{Building a New Node}
    TODO
\clearpage

\section{Decommissioning a Node}
    TODO
\clearpage

\section{Creating Tickets}
    TODO
\clearpage

\section{Checking for Errors}
    \subsection{Monitoring with Icinga}
        \subsubsection{Icinga Basics}
        \subsubsection{Web Interface}
        \subsubsection{Icinga CLI}
    \subsection{Error Logs}
    \subsection{Hardware Errors}
\clearpage


% APPENDIX
\appendix
\section{Node Locations}
\clearpage

\end{document}
